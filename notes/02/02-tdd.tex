We've already seen that it can be useful to write tests for our code.
\\

But what if we wrote the tests \textit{first}? This is the idea behind \textbf{Test Driven Development}.


\section{Red, Green, Refactor}

\section{Git Hooks}

Now that we're test-driven, we should be careful to never commit any code to Git \textit{if} any of our tests are failing.
\\

We can use \textbf{Git Hooks} to guarantee that we can't commit failing code. Hooks are custom bits of code that run at specific points in the Git life-cycle.
\\

We're interested in the \texttt{pre-commit} hook. If this bit of code returns an \textbf{error code} (any number except \texttt{0}) then Git won't allow the commit to run.
\\

When all of your PHPUnit tests succeed you get back a \texttt{0} exit code, but if any fail you'll get back a non-zero exit code. That means all we need to put in our hook is code that runs PHPUnit.
\\

Hooks live in the \texttt{.git/hooks} directory inside your project. So we need to create a file called \texttt{.git/hooks/pre-commit} and then put the following inside:

\begin{minted}{bash}
    #!/bin/bash

    vendor/bin/phpunit --filter Unit
\end{minted}

This is just the command that we've been running manually.
\\

We also need to give this file execution permissions, otherwise it can't be run:

\begin{minted}{bash}
    chmod +x .git/hooks/pre-commit
\end{minted}

Now, whenever you try to commit and code to Git the tests will run and if any of them fail the commit will fail.



\section{Additional Resources}

\begin{itemize}[leftmargin=*]
    \item \href{https://www.freecodecamp.org/news/test-driven-development-what-it-is-and-what-it-is-not-41fa6bca02a2/}{What is Test Driven Development?}
    \item \href{https://www.sandimetz.com/99bottles}{99 Bottles of OOP}
    \item \href{https://git-scm.com/book/en/v2/Customizing-Git-Git-Hooks}{Git Hooks}
\end{itemize}
