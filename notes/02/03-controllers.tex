\section{Routing}

Routing is the process of determining what code should be run based on the given URL. This is a concept you'll come across a lot when developing websites. In fact, we'll come across it again next week when we look at React.
\\

When a user loads a page in a Laravel app, after setting everything up, Laravel has a look to see if the URL matches any of the defined \textbf{routes} (which live in the \texttt{routes} directory). A route represents a specific URL and HTTP method combination. If Laravel finds a match it then runs the code that the route points to.
\\

First, let's add a route to handle \texttt{POST} events sent to \texttt{/articles}. This will point to code that handles creating a new article.

\pagebreak

Add a \texttt{POST} route for \texttt{/articles} to \texttt{routes/api.php}:

\stdinputminted[firstline=3]{php}{02/figures/03/01-route-post.php}

We use the \texttt{\$router} object, which is provided to use by Laravel. It has methods for each of the HTTP methods (e.g. \texttt{\$router->get()}, \texttt{\$router->post()}, etc.). These methods take a URL fragment as the first argument and a controller  method as a second argument (we'll get to these in a second). So the code above says ``If the user makes a \texttt{POST} request to \texttt{/articles}, run the \texttt{store()} method of the \texttt{Articles} controller.



\section{Controllers}

\textbf{Controller} is another term that comes up a lot in software development. A controller's job is to bring together all the other bits of code that make up an app: to \textit{control} what happens. In Laravel, controllers are called by the router: they take the request, do whatever it is that needs doing, and then return a response.
\\

Run the following inside your Vagrant box to create an Articles controller:

\begin{minted}{bash}
    artisan make:controller Articles
\end{minted}

This will create a file (\texttt{app/Http/Controllers/Articles.php}), which contains the boilerplate for a controller.



\section{Route Model Binding}

What if we try to update \texttt{/articles/34849}? Currently we'll get a 500 error as there is no article with ID 34849. But a 500 error is of no use to the client making API requests: it just tells them that you've got a bug in your code.
\\

If an article doesn't exist then we should return a 404 status. The client \textit{can} do something with this as it has a very specific meaning: the thing you requested doesn't exist.
\\

This is such a common issue when building websites/APIs, that Laravel has support for it baked-in. We can use \textbf{Route Model Binding} to ask Laravel to find the appropriate article for us and, if it doesn't find one, return a 404 status automatically.
\\

We do it using PHP type-hinting. Rather than accepting the standard URL parameter argument, we use type-hinting to say to Laravel ``Give us an \texttt{Article} instead''. Laravel does a little bit of magic\footnote{People often complain if frameworks do too much ``magic''. Rails is famous for it. The type-hinting trick is the only case I'm aware of where Laravel can really be accused of it. It's using \href{https://www.culttt.com/2014/07/02/reflection-php}{Reflection} to do this, which is when a programming language looks at its own code and works things out from it. It's probably not something you'll need to use much.} and tries to find the article. If it finds it then it passes in the article object instead of the URL parameter; if it can't find it then it returns a 404 response immediately and doesn't even run the controller code.
\\

To get this working, all we need to do is update all the places in our controller where we accepted the URL parameter \texttt{\$id} to ask for an \texttt{Article} instead. We can then get rid of all the calls to \texttt{Article::find()}, as Laravel will have already done that for us:

\phpinputminted{02/figures/03/02-route-model-binding}



\section{Additional Resources}

\begin{itemize}[leftmargin=*]
    \item \href{https://laravel.com/docs/master/routing}{Routing}
    \item \href{http://laravel.com/docs/master/controllers}{Controllers}
    \item \href{https://laraveldaily.com/how-to-structure-routes-in-large-laravel-projects/}{How to Structure Routes in Large Laravel Apps}
\end{itemize}
