\section{How the Web Works}

\textbf{HyperText Transfer Protocol} is what the World Wide Web is built upon.
\\

Fundamentally it's just sending some text\footnote{This is no longer true with HTTP 2.0, which uses a binary format.}, with a \textbf{well defined format} (or \textbf{standard}), that the browser knows how to interpret into a web page.
\\

By default, a server will listen on \textbf{port 80} for anything that looks like HTTP. If it receives a valid HTTP request then it will return a HTTP response.
\\

In fact, we can manually type in HTTP requests:

\begin{minted}{bash}
    # make a direct connection to the CERN server on port 80
    telnet info.cern.ch 80
\end{minted}

Type in the following exactly and press return twice:

\begin{minted}{http}
    GET /hypertext/WWW/TheProject.html HTTP/1.1
    Host: info.cern.ch
\end{minted}

You should get back a valid HTTP response.
\\

You just connected to a server at CERN, had a chat with it, and asked it to give you back a webpage. Because you said the right thing, it obliged and gave you a response.

\begin{infobox}{The Internet vs. The Web}
    The ``internet'' and the ``world wide web'' are, historically, not the same thing. The internet started off as ARPANet in the late 60s. It was a way of connecting together government computers and universities in a way where it wasn't necessary that every ``node'' in the network needed to be connected directly.
    \\

    \img{12cm}{01/img/arpa-net}{1em}{The internet was so small in 1977 that you could map every computer connected to it on a piece of A4 (\href{http://som.csudh.edu/fac/lpress/history/arpamaps/}{source})}

    Things like e-mail, IRC, newsgroups, and bulletin boards all existed before the web.
    \\

    The ``World Wide Web'' was invented in 1989 by Sir Tim Berners Lee, who was working at CERN at the time. He wanted to create a way to click on a reference in a research paper and be taken directly to the referenced paper - what we'd now call a link. He didn't expect this to have much use outside of academia.
\end{infobox}


\pagebreak


\section{Requests}

An HTTP request, at its most minimal, will generally consist of the following:

\begin{minted}{http}
    GET /api/articles HTTP/1.1
    Host: restful.training
    Accept: application/json
    Accept-Encoding: gzip, deflate
\end{minted}

\begin{enumerate}[leftmargin=*]
    \item \texttt{GET}: the HTTP method
    \item \texttt{/api/articles}: the URI fragment
    \item \texttt{HTTP/1.1}: the HTTP version
    \item \texttt{Host:}: the server hostname header
    \item \texttt{Accept:}: the accept header - what format to return
    \item \texttt{Accept-Encoding:}: the accept encoding header - can we handle gzipped files?
\end{enumerate}

We can see all of this and more for every single request in the Network panel of Developer Tools.


\subsection{Methods}

The \href{https://www.w3.org/Protocols/rfc2616/rfc2616.txt}{HTTP standard} lists a number of \textbf{methods} that can be used:

\begin{itemize}
    \item \texttt{OPTIONS}
    \item \texttt{GET}
    \item \texttt{HEAD}
    \item \texttt{POST}
    \item \texttt{PUT}
    \item \texttt{DELETE}
    \item \texttt{TRACE}
    \item \texttt{CONNECT}
\end{itemize}

Browsers typically only use the \texttt{GET} and \texttt{POST} methods: when you view a webpage you are making a \texttt{GET} request; when you submit a form you are making a \texttt{POST} request.
\\

The other methods largely went unused until RESTful APIs came along.

\section{Responses}

A typical response:\footnote{You can use \texttt{curl -i http://example.com} to see the response for any page.}

\begin{minted}{http}
    HTTP/1.1 200 OK
    Content-Encoding: gzip
    Content-Length: 183
    Content-Type: application/json; charset=UTF-8
    Connection: Closed

    [
        {
            "id": 1,
            "name": "Alice",
            "won": 0,
            "lost": 0,
            "created_at": "2017-07-04 16:03:11",
            "updated_at": "2017-07-04 16:03:11"
        }
    ]
\end{minted}

\begin{itemize}[leftmargin=*]
    \item \texttt{HTTP/1.1}: the HTTP version
    \item \texttt{200 OK}: the HTTP status code and message
    \item \texttt{Content-Encoding:}: the content encoding header
    \item \texttt{Content-Length:}: the content length header
    \item \texttt{Content-Type:}: the content type/character encoding header
    \item \texttt{Connection:}: the connection header
\end{itemize}

Then two line breaks, followed by the body (HTML or JSON).


\pagebreak


\subsection{Status Codes}

The HTTP spec also defines a number of ``status codes'' which have specific meanings:

\begin{itemize}
    \item \texttt{2xx}: Success
        \begin{itemize}
            \item \texttt{200}: OK
            \item \texttt{201}: Created
            \item \texttt{204}: No Content
        \end{itemize}

    \item \texttt{3xx}: Redirect

    \item \texttt{4xx}: Client Error
        \begin{itemize}
            \item \texttt{400}: Bad Request
            \item \texttt{401}: Unauthorised
            \item \texttt{403}: Forbidden
            \item \texttt{404}: Not Found
            \item \texttt{405}: Method Not Allowed
            \item \texttt{422}: Unprocessable Entity
            \item \texttt{429}: Too Many Requests
            \item \texttt{418}: I'm a Teapot\footnote{As per \href{https://tools.ietf.org/html/rfc2324\#section-2.3.2}{RFC 2324 Hyper Text Coffee Pot Control Protocol}: Any attempt to brew coffee with a teapot should result in the error code ``418 I'm a teapot''. The resulting entity body MAY be short and stout.}
        \end{itemize}

    \item \texttt{5xx}: Server Error
\end{itemize}

Client errors are when the user did something wrong (e.g. requested a page that doesn't exist, submitted the wrong information). Server errors are when your server code has done something wrong. Validation, which you looked at earlier, is the process of turning 500 errors (\textit{we} did something wrong) into 400 errors (\textit{you} did something wrong).


\section{Additional Resources}

\begin{itemize}[leftmargin=*]
    \item \href{https://httpstatuses.com}{HTTP Status Codes}
    \item \href{https://developer.mozilla.org/en-US/docs/Web/HTTP/Methods}{MDN: HTTP Methods}
    \item \href{https://www.restapitutorial.com/httpstatuscodes.html}{HTTP Status Codes}
    \item \href{https://twobithistory.org/2018/06/10/birth-of-the-web.html}{The World Wide Web and Its Inventor}
    \item \href{https://www.charlesproxy.com}{Charles Proxy}: view all the HTTP requests on your computer
    \item \href{https://evertpot.com/http/451-unavailable-for-legal-reasons}{451: Unavailable for Legal Reasons}
    \item \href{https://developer.mozilla.org/en-US/docs/Web/HTTP/Status/418}{418: I'm a Teapot}
    \item \href{https://www.youtube.com/watch?v=6iYfvshY4e0}{What is Internet Anyway? An Internaut's Guide}
    \item Try running \texttt{curl https://wttr.in}
\end{itemize}
