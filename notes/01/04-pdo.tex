PDO (``PHP Data Objects'') provides an interface for working with MySQL from PHP.
\\

PDO is an ``abstraction layer'', which means we can actually also use PDO to connect to other types of database such as SQLite and PostgreSQL, without needing to change much of our code.


\begin{infobox}{The history of MySQL with PHP}
    Originally we used the \texttt{mysql} functions to work with MySQL in PHP. However, these functions lead to quite messy code and lack many security features. They are no longer usable in PHP.
    \\

    \texttt{mysqli} was designed to replace \texttt{mysql} - it literally stands for ``MySQL Improved''. It adds useful security features and also an optional object-oriented API.
    \\

    More recently it's become common to use other SQL databases such as PostgreSQL and SQLite. It's desirable to be able to use any of these from PHP. So \texttt{PDO} was introduced as a way to support other databases in an object-oriented style.
\end{infobox}

\section{Connecting}

We need to first create an object representing a connection to the database:

\begin{minted}{php}
    $pdo = new PDO('mysql:host=localhost;dbname=database', 'username', 'password');
\end{minted}

Once we've created it we can use the \texttt{\$pdo} object to do things with our database.


\section{Running a Query}

Running a query consists of two parts. First we generate the query, then we run it. Until we run it, nothing actually happens


\section{MySQL Injection}



\section{Additional Resources}

\begin{itemize}[leftmargin=*]
    \item \href{http://php.net/manual/en/mysqlinfo.api.choosing.php}{Choosing a PHP API}: The differences between \texttt{mysql}, \texttt{mysqli}, and \texttt{PDO}.
\end{itemize}
