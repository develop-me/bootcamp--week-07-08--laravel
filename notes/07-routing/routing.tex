Routing is the process of determining what code should be run based on the given URL. This is a concept you'll come across a lot when developing websites. In fact, we'll come across it again when we look at React.
\\

When a user loads a page in a Laravel app, after setting everything up, Laravel has a look to see if the URL matches any of the defined \textbf{routes} (which live in the \texttt{routes} directory). A route represents a specific URL. If Laravel finds a match it then runs the code that the route points to.
\\

If you have a look in \texttt{routes/web.php}, you'll see that it already contains a route:

\begin{minted}{php}
    <?php

    use Illuminate\Support\Facades\Route;

    Route::get('/', function () {
        return view('welcome');
    });
\end{minted}

We use a static method of the \texttt{Route} class to say that if the user visits \texttt{/} (the \textbf{root}\footnote{Yes, it's the root route} URL – e.g. the homepage of our site) we should return a \textbf{view} called ``welcome''. This is why, if you visit your Laravel site, you'll see the \texttt{welcome.blade.php} template. But if you go to any other URL you'll get a 404.
\\

Let's add another route for an ``About'' page:

\begin{minted}{php}
    Route::get('/about', function () {
        return view('about');
    });
\end{minted}

Next, create a template \texttt{resources/views/about.blade.php}:

\inputminted{html}{07-routing/figures/01-about.blade.php}

Now, visit \texttt{/about} if your browser and you should see the new page.



\section{URL Parameters}

Next, let's add a route for an article.
\\

First create the file \texttt{resources/views/article.blade.php}:

\inputminted{html}{07-routing/figures/02-article.blade.php}

And then add another route to \texttt{routes/web.php}:

\begin{minted}{php}
    Route::get('/articles/{id}', function () {
        return view('article');
    });
\end{minted}

Now if you visit the URL \texttt{/articles/1} in your browser, you should see the article view that you just created. It will also work if you go to \texttt{articles/2}. In fact it will work for \texttt{articles/anything}. That's because the \texttt{\{id\}} bit is a \textbf{URL parameter}: a placeholder for \textit{any} value. We'll look at how we can work with this value later.

\section{Additional Resources}

\begin{itemize}[leftmargin=*]
    \item \href{http://laravel.com/docs/master/routing}{Laravel: Routing}
\end{itemize}
