There are many ways to deploy your Laravel API; many of them highly automated, efficient and pain free.
\\

We won't be using any of those.
\\

Even though it's a slightly more involved process, we're going to use an AWS Linux server for the following reasons:

\begin{itemize}
    \item It's free (up to a point)
    \item It looks and feels like your Vagrant box
    \item It can be setup to use the same stack as your Vagrant box
\end{itemize}

AWS (Amazon Web Services) provides a dizzying array of cloud-computing services on a pay as you go basis. These include computing, storage, networking, and databases. The most popular of these include Elastic Cloud Computing (EC2) and Simple Storage Service (S3).
\\

A good proportion of the internet runs on AWS. It powers Netflix, Adobe, Airbnb, IMDb, Slack, and many other major websites.



\section{Working With EC2}

Elastic Cloud Computing is a service that allows you to spin up virtual computers to deploy your applications, rather than having to invest in hardware. They are ``elastic'' as they can be configured to scale up or down depending on traffic.
\\

Before you begin you'll need to \href{https://portal.aws.amazon.com/gp/aws/developer/registration}{sign up for an AWS account}.  Once you've got an account click on ``Sign in to the Console''.


\subsection{Launching the Server}

To launch your server just follow these simple and easy to remember steps:

\begin{itemize}
    \item Login to AWS, go to ``EC2'', and make sure you're on the ``EU London'' region (in the top right).
    \item Launch Instance
    \item Choose: ``Ubuntu Server 18.04 LTS (HVM), SSD Volume Type''
    \item Choose: ``T2.micro''
    \item Go to ``5. Add Tags'' in the tabs and add the tag:
        \begin{itemize}
            \item \textbf{Key}: \texttt{Name}
            \item \textbf{Value}: \texttt{my-server-name}
        \end{itemize}
        This is so you can identify it in the dashboard.

    \item Go to the ``6. Configure Security Group'' tab
        \begin{itemize}
            \item Add rule and select HTTPS.
            \item Add rule and select HTTP.
        \end{itemize}
         This enables the sending and receiving of data via HTTPS/HTTP.
    \item Click ``Launch''
    \item On the final screen ``Create a new key pair'' and download the \texttt{pem} file
    \item Go back to the EC2 dashboard and you should see this server loading up
\end{itemize}


\section{SSH}

SSH (Secure SHell) enables secure system administration and file transfers over insecure networks. You've already used it in when you added an SSH key to Git.
\\

SSH takes advantage of public-key cryptography. You create a ``public'' SSH key which lives on the server. When you try to connect to the server it uses the public key to encrypt a message, which can only be decrypted with the ``private'' SSH key, which lives on your machine (\textbf{Never share a private SSH key!}). The message contains a ``symmetric'' encryption key that can then be used for the remainder of the communications. If the user doesn't have the right private key then they can't decrypt the message and they won't be able to login.\footnote{This is a massive simplification, but the general gist is right}
\\

Once a secure connection is created you can use SSH to run commands on a remote machine (usually a server) just as if you are running them on your local machine.


\begin{infobox}{Public-Key Cryptography}
    Before the mid-70s the only way to encrypt a message was for both the sender and receiver to agree on an encryption key in advance (\textit{symmetric key cryptography}). This has the obvious issue that at some point the encryption key needs to be shared and if someone intercepts it then the whole system is somewhat pointless.
    \\

    In 1977\footnotemark Ron Rivest, Adi Shamir, and Leonard Adleman (RSA) came up with a system whereby a ``public'' key could be shared, but this can only be used to \textit{encrypt} the data. To \textit{decrypt} the data you would need the ``private'' key.
    \\

    This takes advantage of the idea that certain mathematical operations are easy to do in one direction but not the other. For example, it's easy to multiply together two large prime numbers:

    \begin{minted}{text}
         2833419889721787128217599 × 195845982777569926302400511
    \end{minted}

    However, given only the result of the multiplication, it's very hard to work out what those two numbers were (if they are both prime then it's guaranteed there will only be one correct answer):

    \begin{minted}{text}
        554913902924077201755461386242561041733898456793089
    \end{minted}

    The public key is like the result of the multiplication and the private key is knowing which two numbers you multiplied to get there. So we can share the public key (the result) without worrying about anyone working out what the private key is.
    \\

    The actual process involves \textit{much} larger numbers than the ones above (1,024 bit encryption, considered quite weak, involves numbers with 300+ digits). It also involves modulus (\texttt{\%}), because modulus is great.
\end{infobox}

\footnotetext{It was independently invented by Clifford Cocks in 1973, but he was working for GCHQ at the time, so this wasn't public knowledge until 1997}



\subsection{Accessing the Server}

You should copy the \texttt{pem} file into the \texttt{\textasciitilde/.ssh} directory. Run this inside your \texttt{Downloads} folder:

\begin{minted}{bash}
    cp my-server-key.pem ~/.ssh
\end{minted}

Next, we need to make sure no one else on your computer can use the file (remember, anyone with access to this file can access your server):

\begin{minted}{bash}
    chmod 400 ~/.ssh/my-server-key.pem
\end{minted}

For an Ubuntu instance the username is \texttt{ubuntu} and the domain can be found by clicking on your instance in the EC2 dashboard and copying the Public DNS (IPv4). It should something look like this:

\begin{minted}{bash}
    ec2-3-8-39-24.eu-west-2.compute.amazonaws.com
\end{minted}

Next, we need to tell SSH to use the key we just created whenever it tries to connect to that server. In \texttt{\textasciitilde/.ssh/config}:

\begin{minted}{text}
    Host ec2-3-8-199-58.eu-west-2.compute.amazonaws.com
      User ubuntu
      IdentityFile ~/.ssh/my-server-key.pem
      IdentitiesOnly yes
\end{minted}

Make sure you point to \textit{your} \texttt{pem} file and the correct IPv4 address.
\\

Now, to login to your server run:

\begin{minted}{bash}
    ssh ubuntu@ec2-3-8-199-58.eu-west-2.compute.amazonaws.com
\end{minted}

That's quite a lot to remember, to let's update our \texttt{\textasciitilde/.ssh/config} with an alias:

\begin{minted}{text}
    Host aws ec2-3-8-199-58.eu-west-2.compute.amazonaws.com
      HostName ec2-3-8-199-58.eu-west-2.compute.amazonaws.com
      User ubuntu
      IdentityFile ~/.ssh/my-server-key.pem
      IdentitiesOnly yes
\end{minted}

The \texttt{aws} bit after \texttt{Host} adds an alias for us so that we don't need to type out the full username and address. Now we can just do:

\begin{minted}{bash}
    ssh aws
\end{minted}

Much easier to remember!



\section{Setting Up the Server}

You will need to install the following software packages on the server:

\begin{itemize}
    \item \textbf{Nginx}: HTTP server software
    \item \textbf{MySQL}: database management system
    \item \textbf{PHP}: programming language
    \item \textbf{Git}: file management
    \item \textbf{Composer}: PHP package manager
    \item \textbf{SSH Keys}: for authentication
\end{itemize}

First update \texttt{apt-get}, used for installing Linux packages, so that it has the most up to date package versions:

\begin{minted}{bash}
    sudo apt-get update -y
\end{minted}

First, let's install Nginx:

\begin{minted}{bash}
    sudo apt-get install -y nginx
\end{minted}

Once this is installed, visit your IPv4 address in the browser. For example:

\begin{minted}{text}
    http://ec2-3-8-125-57.eu-west-2.compute.amazonaws.com
\end{minted}

You should see a ``Welcome to nginx!'' message.
\\

Now, let's install PHP,  MySQL and Git:

\begin{minted}{bash}
    sudo apt-get install -y php-fpm mysql-server php-mysql php-mbstring php-dom zip git
\end{minted}

And finally install Composer:

\begin{minted}{bash}
    # navigate to the home directory
    cd ~
    # download and run the composer install script
    sudo curl -sS https://getcomposer.org/installer | sudo php
    # move the composer.phar file to the right location
    sudo mv composer.phar /usr/local/bin/composer
    # setup a shortcut so that you can use it in any directory
    sudo ln -s /usr/local/bin/composer /usr/bin/composer
    # own the composer directory
    sudo chown -R ubuntu ~/.composer
\end{minted}

You can check Composer has installed properly by running \texttt{composer}.


\subsection{Ownership}

By default, your files are served from the directory: \texttt{/var/www/html}. This is like your \texttt{public} directory in a Vagrant box.
\\

Try creating a file in \texttt{/var/www/html}:

\begin{minted}{bash}
    cd /var/www/html
    touch index.html
\end{minted}

You should get the message \texttt{touch: cannot touch `index.html': Permission denied}.
\\

This is because, by default, all files on the server are owned by the \texttt{root} user.
\\

Let's take ownership of this directory: so that we can make changes:

\begin{minted}{bash}
    sudo chown -R ubuntu /var/www/
\end{minted}

To check it's working, create an index page on your server:

\begin{minted}{bash}
    touch index.html
\end{minted}

Then open the file in a code editor. We're editing it on the server, so we'll need to use a command-line text editor. Nano is probably the least baffling one to start with:

\begin{minted}{bash}
    nano index.html
\end{minted}

In Nano add some basic HTML:

\begin{minted}{html}
    <h1>Hello, world!</h1>
\end{minted}

Then press \texttt{Ctrl+o} to save and then \texttt{Ctrl+x} to exit.
\\

Once again, check your server in the browser. You should now see ``Hello, World!'', as Nginx is serving up your \texttt{index.html} file.
\\

Finally, remove the \texttt{index.html} file and the one Nginx created:

\begin{minted}{bash}
    rm index*.html
\end{minted}


\section{Deployment}

We're going to deploy files to the server using Git. We'll setup a new set of SSH keys, this time on the server. We'll give the public key to GitHub and then we'll be able to access our GitHub repositories from the server.


\subsection{Generating Server SSH Keys}

Create a key on the server (hit \texttt{Enter} at all prompts):

\begin{minted}{bash}
    ssh-keygen -t rsa -b 4096 -C "<your_email_address>"
\end{minted}

Substitute the email for the best email to contact you on. This helps your successor when they come to clean up random keys left on the server.
\\


You should now be able to see this key-pair in the directory:

\begin{minted}{bash}
    cd ~/.ssh && ls
\end{minted}

Start \texttt{ssh-agent}:

\begin{minted}{bash}
    eval "$(ssh-agent -s)"
\end{minted}

Then add the private key to \texttt{ssh-agent}:

\begin{minted}{bash}
    ssh-add ~/.ssh/id_rsa
\end{minted}

Echo out your public key:

\begin{minted}{bash}
    cat ~/.ssh/id_rsa.pub
\end{minted}

Then select the output and copy to the clipboard.
\\

In GitHub, go to top right profile picture, click ``Settings'', in the side bar click ``SSH and GPG Keys''. Then click on ``New SSH Key'' and paste in the key. Give it a sensible title about where this lives, such as the name of the server.
\\

You should now be able to connect to Git from the server.


\subsection{Copy Your Repository to the Server}

On your repository, go to clone or download and, as usual, copy the code (ensure it says clone with SSH):

\begin{minted}{bash}
    cd /var/www
    git clone git@github.com:your-name/repo-name.git
    cd repo-name
    composer install
\end{minted}



\subsection{Setting Up Laravel}

Your Laravel project is served out of the \texttt{/var/www/<repo-name>/public} directory, whereas your server is setup to look for an index page in \texttt{/var/www/html}. So, we need to tell the server where the project root is.
\\

Open the Nginx config file in a text editor:\footnote{It's good practice to use site specific configuration files, but we'll just edit the default one for now}

\begin{minted}{bash}
    sudo nano /etc/nginx/sites-available/default
\end{minted}

First, add \texttt{index.php} to the acceptable index files:

\begin{minted}{nginx}
    index index.php index.html index.htm;
\end{minted}

Scroll down to \texttt{root} and change it to:

\begin{minted}{nginx}
    root /var/www/<repo-name>/public;
\end{minted}

We'll also need to update the \texttt{location} entries:

\begin{minted}{nginx}
    # redirect everything to index.php
    location / {
        try_files $uri $uri/ /index.php?$query_string;
    }

    # run php with php-fpm
    location ~ \.php$ {
        include snippets/fastcgi-php.conf;
        fastcgi_pass unix:/var/run/php/php7.2-fpm.sock;
    }
\end{minted}

Now reboot Nginx:

\begin{minted}{bash}
    sudo nginx -s reload
\end{minted}

Finally, give Nginx permission to use the directory:

\begin{minted}{bash}
    # give permission to nginx for all future files
    sudo setfacl -Rdm u:www-data:rwX /var/www/

    # give permission to nginx for all current files
    sudo setfacl -Rm u:www-data:rwX /var/www/
\end{minted}

\begin{infobox}{File Access Control Lists}
    The permissions system of Unix (and thus Linux) was based on the idea that a single user ``owned'' directories and files on a computer. This was a significant security upgrade in the 70s, but for more modern uses it's a bit lacking.
    \\

    \textbf{File Access Control Lists} let us set permissions on a file/directory for as many users as we like. This is very useful, but as always, with great power comes great responsibility.
\end{infobox}

Once again check the page in your browser. The Nginx test message should have disappeared and everything should be broken. Congratulations, you have made progress!
\\

You're getting an error because you are missing the \texttt{.env} file that contains your Laravel configuration. Copy the example file over:

\begin{minted}{bash}
    cd /var/www/<repo-name>

    # copy the example .env file
    cp .env.example .env

    # generate a key
    php artisan key:generate
\end{minted}

Check the page in your browser. The page should now display a \textit{Laravel} error message saying that we've not setup a MySQL database yet.


\subsection{Setting Up the Database}

First, configure MySQL and setup a password:

\begin{minted}{bash}
    sudo mysql_secure_installation
\end{minted}

Press enter when asked about the current password. Type \texttt{Y} to set a new password and enter and re-enter to confirm, then answer \texttt{Y} to all further questions.
\\

Now we can access MySQL:

\begin{minted}{bash}
    sudo mysql
\end{minted}

And create a new user/database:

\begin{minted}{mysql}
    # create a new user - pick a secure password
    CREATE USER 'laravel'@'localhost' IDENTIFIED BY '<password>';

    # create a new database
    CREATE DATABASE laravel;

    # grant the user access to the database
    GRANT ALL PRIVILEGES ON laravel.* TO 'laravel'@'localhost';

    # quit MySQL
    exit;
\end{minted}

There are plenty of \href{https://generatepasswords.org}{online password generators} if you want something secure.

\subsection{Configure Laravel}

Update your \texttt{.env} file with the details of your database (the name of the database and password you created as well as changing user to root):

\begin{minted}{bash}
    nano .env
\end{minted}

Populate the database details (using the password you picked just now):

\begin{minted}{ini}
    DB_DATABASE=laravel
    DB_USERNAME=laravel
    DB_PASSWORD=<password>
\end{minted}

Now, use \texttt{artisan} to setup the correct structure for the database:

\begin{minted}{bash}
    php artisan migrate
\end{minted}

Success, everything is setup! You should now be able to make calls to your API.


\section{HTTPS \& SSL Certificates}

\textbf{HTTP Secure} (HTTPS) encrypts all information sent between the client and the server and should \textit{always} be used if the site receives any personal data or uses any sort of authentication.
\\

Nowadays it is considered good practice to use HTTPS on \textit{all} websites, as it can prevent a number of security issues.
\\

First step, we'll need to set up a real domain to point to our server. If you don't have one to use ask the instructor to set you up a \texttt{something.developme.space} one.
\\

Now update the Nginx config to use this domain, edit the config file:

\begin{minted}{bash}
    sudo nano /etc/nginx/sites-available/default
\end{minted}

Change the \texttt{server\_name}:

\begin{minted}{nginx}
    server_name <YOUR.DOMAIN>;
\end{minted}

Then restart Nginx:

\begin{minted}{bash}
    sudo nginx -s reload
\end{minted}

Next, we install \texttt{certbot} on our AWS EC2 instance, which we can use to have SSL certificates issued for us, with:

\begin{minted}{bash}
    sudo apt-get install -y certbot python-certbot-nginx
\end{minted}

Then we issue a certificate:

\begin{minted}{bash}
    sudo certbot --nginx
\end{minted}

Read the instructions carefully as you go along.
\\

You'll need to provide \texttt{certbot} with an email address so that it can warn you if a certificate is due to expire. It's up to you whether to sign-up for their email newsletter or not.
\\

When offered the choice, select the relevant domain name from the list (there will probably only be one option). When offered whether to redirect traffic pick \texttt{2: Redirect}, this will ensure that all traffic goes via HTTPS.
\\

Now if you go to your website again it should automatically send you to an HTTPS version.
\\

Certbot will even automatically generate new certificates for you every 90 days - assuming you keep your instance running that long.


\section{Terminating an Instance}

AWS doesn't give you free hosting forever, so at some point you may want to terminate your instance, otherwise it will start charging you.
\\

To terminate an instance:

\begin{itemize}
    \item Go to "EC2" in the AWS Console
    \item Click on "Instances" in the left-hand menu
    \item Select the EC2 Instance you want to terminate
    \item Click on the "Actions" button (above the instances list)
    \item Select "Instance State" then "Terminate"
    \item Agree to terminate the instance
\end{itemize}

Terminating an instance will delete the server including any files and setup on it. Make sure you definitely want to do this!


\section{Additional Resources}

\begin{itemize}[leftmargin=*]
    \item \href{https://www.digitalocean.com/community/tutorials/how-to-install-linux-nginx-mysql-php-lemp-stack-ubuntu-18-04}{Digital Ocean: How To Install Linux, Nginx, MySQL, PHP (LEMP stack) on Ubuntu 18.04}
    \item \href{https://www.digitalocean.com/community/tutorials/how-to-install-and-configure-laravel-with-lemp-on-ubuntu-18-04}{How to Install and Configure Laravel with LEMP on Ubuntu 18.04}
    \item \href{https://docs.aws.amazon.com/AWSEC2/latest/UserGuide/terminating-instances.html}{Terminating an EC2 Instance}
\end{itemize}
