\textbf{Application Programming Interface}s allow programs to talk to one another\footnote{"API" is sometimes also used to refer to the methods on objects, e.g. the \texttt{map} method of an array or the \texttt{.getElementById()} method of DOM nodes}.
\\

But, programs are stupid: they can only deal with things that they're expecting.
\\

So an API has to be \textbf{well-defined}: how we make requests to it and what it gives us back should be \textit{predictable} (and preferably documented).


\section{RESTful}

We can design our API however we like, but if we stick to certain conventions it will be much easier to use the API from other programs.
\\

The most popular way to design an API nowadays\footnote{It's looking like GraphQL might become the new standard in a few years} is using \textbf{RESTful}\footnote{\textbf{RE}presentational \textbf{S}tate \textbf{T}ransfer - don't worry if you can't remember that, I just looked it up}
\\

The idea of RESTful is that we send our data as JSON using HTTP and use the already established HTTP methods/status codes.
\\

Because most programming languages have built in functionality for working with HTTP and JSON, using RESTful APIs means that a program written in one language can talk to a program written in another language without any problems.


\pagebreak


\textbf{Key Ideas}

\begin{itemize}
    \item Sent using HTTP
        \begin{itemize}
            \item HTTP headers
            \item HTTP methods
            \item HTTP status codes
        \end{itemize}
    \item Use URLs for different resources
    \item Use JSON to send/receive data
\end{itemize}


\section{JSON}

JSON is a way of storing and transferring data structures. It is used by most APIs on the internet. And it's based on JavaScript (it stands for \textbf{JavaScript Object Notation}) - so if you understand JS objects/arrays, you can understand JSON.

\begin{minted}{json}
    {
        "people": [{
            "name": "Alice",
            "age": 52,
            "address": {
                "address1": "221b Baker Street",
                "city": "London"
            }
        }, {
            "name": "Bob",
            "age": 34,
            "address": {
                "address1": "13 Morse Street",
                "city": "Exeter"
            }
        }]
    }
\end{minted}

JSON \textbf{is not} JavaScript: JavaScript is a programming language that gets interpreted into machine code and runs, JSON is a data format for representing data so that it can be stored/transferred.
\\

JSON has to be understood by programming languages other than JavaScript which means it is stricter: you can't have stray commas and property keys \textit{have} to be wrapped with \textbf{double quotes}.




\section{Methods}

With RESTful the HTTP methods are given specific meanings:
\\

\begin{tabularx}{\textwidth}{| l | X |}
    \hline
    \textbf{Method} & \textbf{Meaning} \\
    \hline
    \texttt{GET} & displaying data, shouldn't change anything (e.g. the database)\\
    \texttt{POST} & creating new things\\
    \texttt{PUT} & updating things\\
    \texttt{PATCH}\footnote{Added to the spec in 2010} & updating parts of things\\
    \texttt{DELETE} & deleting things\\
    \hline
\end{tabularx}


\section{Status Codes}

Some status codes are also given specific meanings:
\\

\begin{tabularx}{\textwidth}{| c | l | X |}
    \hline
    \textbf{Status Code} & \textbf{Name} & \textbf{Meaning} \\
    \hline
    \texttt{201} & Created & A new item has been created \\
    \texttt{422} & Unprocessable Entity & Validation error \\
    \hline
\end{tabularx}



\section{Naming Routes}

By convention we use something like the following for naming our routes:
\\

\begin{tabularx}{\textwidth}{| l | l | X |}
    \hline
    \textbf{Method} & \textbf{Example URL} & \textbf{Meaning} \\
    \hline
    \texttt{GET} & \texttt{/articles} & Show all articles\\
    \texttt{POST} & \texttt{/articles} & Create new article\\
    \texttt{GET} & \texttt{/articles/1} & Show article with ID 1\\
    \texttt{PUT} / \texttt{PATCH} & \texttt{/articles/1} & Update article with ID 1\\
    \texttt{DELETE} & \texttt{/articles/1} & Delete article with ID 1\\
    \texttt{GET} & \texttt{/articles/1/comments} & Show comments for article with ID 1\\
    \hline
\end{tabularx}

\par\bigskip

Notice that we use the pluralised version of names in the URL.


\pagebreak


\section{RESTful Architecture}

With a traditional website everything comes from the same server: the database, back-end code, and front-end code.
\\

An \textbf{API driven} web app splits the database and back-end from the front-end. The back-end code can be on a server in one bit of the world and the front-end code can be running on another server or an app.
\\

This means that you only have to write the main bit of your app once in the back-end: the various front-ends are just UIs using API requests to do everything.
\\

\img{10cm}{13-http-apis/img/api-01}{1em}{A web app talks to the API using HTTP requests}
\img{10cm}{13-http-apis/img/api-02}{1em}{A completely separate iOS app can use the same API}
\img{10cm}{13-http-apis/img/api-03}{1em}{And an Android app can use the same API too}


\pagebreak

\section{Postman}

\href{https://www.getpostman.com}{Postman} is a free app designed for testing out RESTful APIs.\footnote{For Mac there is an alternative app called \href{https://paw.cloud}{Paw}, which is much better, but not free.}

\img{14cm}{13-http-apis/img/postman}{0em}{Postman}

APIs are not designed for people, so when you're testing them with Postman you need to get used to writing/reading JSON. Remember, it's just a slightly stricter JS object/array syntax.

\subsection{Postman Tips}

\begin{itemize}[leftmargin=*]
    \item Make sure you set the \texttt{Accept: application/json} header

        \img{12cm}{13-http-apis/img/headers}{1em}{\textbf{Always} set this header}

    \pagebreak

    \item Make sure you use ``raw'' and then ``JSON (application/json)'' for the Body

        \img{12cm}{13-http-apis/img/raw}{1em}{Make sure you use JSON for the Body}

    \item Duplicate tabs rather than creating new ones or editing the one you already have

        \img{9cm}{13-http-apis/img/duplicate}{1em}{It's easier to duplicate tabs}

    \item Use ``Two Pane View'' (\texttt{Cmd + Alt + V} or icon in the bottom right)

        \img{5cm}{13-http-apis/img/split}{1em}{In two pane view you don't have to scroll to see the response}

    \item Make sure you don't have any errors in your JSON before submitting data

        \img{12cm}{13-http-apis/img/error}{1em}{If there's an error in your JSON you'll get confusing validation errors}

    \item Use Collections to save routes you're likely to use more than once
\end{itemize}



\section{Additional Resources}

\begin{itemize}[leftmargin=*]
    \item \href{https://www.howtogeek.com/343877/what-is-an-api/}{What is an API?}
    \item \href{https://restfulapi.net}{REST API Tutorial}
    \item \href{https://www.smashingmagazine.com/2018/01/understanding-using-rest-api/}{Understanding and Using REST APIs}
    \item \href{https://any-api.com}{Any API}: documentation and interactive consoles for tonnes of public APIs
    \item \href{https://webapplog.com/graphql/}{Why GraphQL is taking over APIs}
    \item \href{https://ponyfoo.com/articles/graphql-in-depth-what-why-and-how}{GraphQL in Depth: What, Why, and How}
\end{itemize}
