Routing is the process of determining what code should be run based on the given URL. This is a concept you'll come across a lot when developing websites. In fact, we'll come across it again next week when we look at React.
\\

When a user loads a page in a Laravel app, after setting everything up, Laravel has a look to see if the URL matches any of the defined \textbf{routes} (which live in the \texttt{routes} directory). A route represents a specific URL and HTTP method combination. If Laravel finds a match it then runs the code that the route points to.
\\

First, let's add a route to handle \texttt{GET} requests sent to \texttt{/articles}. This will point to code that handles show a list of articles.

\pagebreak

Add a \texttt{POST} route for \texttt{/articles} to \texttt{routes/web.php}:

\phpinputminted{05-routing/figures/01-route-post}

We use the \texttt{\$router} object, which is provided to use by Laravel. It has methods for each of the HTTP methods (e.g. \texttt{\$router->get()}, \texttt{\$router->post()}, etc.). These methods take a URL fragment as the first argument and a controller  method as a second argument (we'll get to these in a second). So the code above says ``If the user makes a \texttt{GET} request to \texttt{/articles}, run the \texttt{index()} method of the \texttt{Articles} controller.
