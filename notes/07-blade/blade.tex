Laravel has its own \textbf{templating language} called ``Blade''.
\\

A templating language lets you easily combine the data in your app with some other format to create a final document. In the case of Blade, it lets us easily generate the final HTML output of our app, making it easy to:
\\

\begin{itemize}
    \item Reuse repeated markup
    \item Display data
    \item Conditionally show content
    \item Loop over Collections
\end{itemize}

Blade templates have a \texttt{.blade.php} extension and they live inside the\texttt{resources/views} directory.
\\

If you have a look at \texttt{resources/views/welcome.blade.php} you'll see the initial ``Larvel'' page that you get whenever you first create a Laravel project.\footnote{In the next section we'll look at \textbf{routing} which allows us to change which view we get} We're going to be gradually changing this file show a list of articles from our database.
\\

First, let's strip it back to basics:

\code{resources/views/welcome.blade.php}{html}{07-blade/figures/01-basic.blade.php}

Now reload the site.
\\

We've got a title and a some filler text for a blog post. And it looks like a bag of arse. Bootstrap to the rescue. In your \texttt{<head>}:

\begin{minted}[fontsize=\scriptsize]{html}
    <link
      rel="stylesheet"
      href="https://stackpath.bootstrapcdn.com/bootstrap/4.3.1/css/bootstrap.min.css"
      integrity="sha384-ggOyR0iXCbMQv3Xipma34MD+dH/1fQ784/j6cY/iJTQUOhcWr7x9JvoRxT2MZw1T"
      crossorigin="anonymous"
    />
\end{minted}

And your \texttt{<body>}:

\code{resources/views/welcome.blade.php}{html}{07-blade/figures/02-styled.blade.php}

What a lot of \texttt{class}es! But it looks much nicer if you load it in the browser.



\section{Extending}

You'll probably notice that some of the material in our template is fairly generic markup that we'd want to use on every page of our site (e.g. the \texttt{<head>} and \texttt{<nav>}), whereas some of it would only be relevant for the homepage (e.g. the blog post list).
\\

Blade lets us separate these two different aspects. We'll create a file called \texttt{app.blade.php} for the markup that's the same on every page:

\code{resources/views/app.blade.php}{html}{07-blade/figures/03-app.blade.php}

And then update \texttt{welcome.blade.php} to use it:

\code{resources/views/welcome.blade.php}{html}{07-blade/figures/04-welcome.blade.php}

There are two key things:

\begin{itemize}
    \item \texttt{@extends}: we are extending the \texttt{app.blade.php} file (we don't need to include the file extension)
    \item \texttt{@section("content")}: we have created a named section called ``content'', which the \texttt{app} template \texttt{yield}s
\end{itemize}

Now, reload the site and everything should be as before. But we've modularised our code to make it much more reusable.


\section{Partials}

Sometimes it can be useful to split up your templates further. For example, as our app gets more complicated the \texttt{app.blade.php} file might get quite large. We could break it up into \textbf{partials}: small bits of template that we can include from another file.

\pagebreak

Let's create a file \texttt{resources/views/_partials/nav.blade.php} and move our navigation markup into it:

\code{resources/views/_partials/nav.blade.php}{html}{07-blade/figures/05-nav.blade.php}

Now, in \texttt{app.blade.php} we can use this file:

\begin{minted}{html}
    <div class="container">
        @include("_partials/nav")

      <main class="mt-4">
\end{minted}

You'll notice that we're not extending or using sections in the partial. That's because a partial is always \texttt{@include}d from another file.


\begin{infobox}{Dotty}
    If you read the Laravel docs you'll often see something like:

    \begin{minted}{html}
        @include("_partials.nav")
    \end{minted}

    On a Linux server, using a \texttt{.} instead of a \texttt{/} does exactly the same thing.
\end{infobox}



\section{Loops \& Moustaches}

It's often useful to repeat a certain bit of markup multiple times. Let's use a loop to update our articles list to show multiple articles.
\\

First, make sure you've got multiple articles added to your database. Use \texttt{artisan tinker} to add some more if you've not got many.
\\

Now update \texttt{welcome.blade.php} with the following:

\code{resources/views/welcome.blade.php}{html}{07-blade/figures/06-loop.blade.php}

We use the \texttt{@foreach} directive to loop over all of our articles. Each one gets passed in as \texttt{\$article}. We can then use \textbf{moustaches} (double curly-braces) to interpolate PHP values into our template.

\section{Passing Data}

Let's put the code for a single list item into its own partial.
\\

In \texttt{resources/views/_partials/articles/list-item.blade.php}:

\code{resources/views/_partials/articles/list-item.blade.php}{html}{07-blade/figures/07-list-item.blade.php}

Now we can include this from \texttt{welcome.blade.php} and pass in the \texttt{\$article}:

\begin{minted}{html}
    @foreach (App\Article::all() as $article)
      {{ /* pass-through $article as "article" */ }}
      @include("_partials/articles/list-item", ["article" => $article])
    @endforeach
\end{minted}

We can pass an associative-array as the second argument to \texttt{@include} and that will then be available inside the partial with the variable name that we gave it (\texttt{"article"} in this case).

\section{Additional Resources}

\begin{itemize}[leftmargin=*]
    \item \href{http://laravel.com/docs/master/blade}{Laravel: Blade Templates}
\end{itemize}
