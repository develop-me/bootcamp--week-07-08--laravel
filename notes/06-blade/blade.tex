Laravel has its own templating language called ``Blade'', which allows us to easily render HTML.
\\





\section{Layouts}

The front-end views you see in your Laravel site are produced from Blade template files.
\\

The files have \texttt{.blade.php} file extension and are stored in the \texttt{resources/views} folder.
\\

\section{Extending a Layout}

Templating is often done by setting up a master layout and then extending, or overriding parts of that template to create child templates.
\\

This is done by using \texttt{@extends} within the child layout file.
\\

For example \texttt{@extends('layouts.app')} is used to extend a layout called \texttt{layouts.app.blade.php}.


\section{Section and Yield Directives}

Sections of layout can be defined in a master template and overridden in a child template.
\\

A \texttt{@section} that ends with \texttt{@endsection} directive will only define a section.
\\

A \texttt{@section} that ends with \texttt{@show} directive will define and immediately yield (show) the section.
\\

In this way we can have default content in the master, and override it in the child template as below:
\\

\textbf{Master layout \texttt{master.blade.php}}

\begin{minted}{php}
    @section('content')
        <h1>My Default Content<h1>
    @show
\end{minted}

\textbf{Child layout \texttt{child.blade.php}}
\begin{minted}{php}
    @extends('master')
    @section('content')
        <h1>This will override the default. Ha!<h1>
    @endsection
\end{minted}

\section{Partials}

Template parts can be made more modular, for reuse or to keep things neater, using partial template files.
\\

The partials can then be included back into a layout or multiple layouts with \texttt{@include}.
\\

\section{Additional Resources}

\begin{itemize}[leftmargin=*]
    \item \href{http://laravel.com/docs/master/blade}{Blade Templates}
\end{itemize}
