\textbf{Unit Testing} is when we test small parts of our code to check that it works as we expect. This can save us having to dump bits of code to see what's going on.
\\

Because unit tests are bits of code, we can run them as often as we like. This is much more efficient than manually testing/re-testing your code.
\\

Laravel makes it very easy to unit test code. All our unit tests live in the \texttt{tests/Unit} directory. You can see an example test called \texttt{ExampleTest.php} already in there.

\section{Running Tests}

To run all of your unit tests run the following:

\begin{minted}{bash}
    vendor/bin/phpunit --testsuite Unit
\end{minted}

You should get a message saying:

\begin{minted}{text}
    OK (1 test, 1 assertion)
\end{minted}

If you don't include the \texttt{--testsuite Unit} bit, it will also run \textbf{Feature Tests}, which let you test broader functionality, but we're not interested in this at the moment.
\\

You can now delete the \texttt{tests/Unit/ExampleTest.php} file, as we'll create our own tests next.


\section{Generating Tests}

Laravel can automatically generate unit tests for us:

\begin{minted}{bash}
    artisan make:test ArticleTest --unit
\end{minted}

There should now be a file called \texttt{ArticleTest} in the unit tests directory.


\section{Writing Tests}

Let's have a look at the test class:

\php{tests/Unit/ArticleTest.php}{06-unit-testing/figures/01-ArticleTest}

Test class names need to \textit{end} with \texttt{Test} and to extend \texttt{TestCase} – this is a class provided by PHPUnit that does most of the heavy-lifting for us.
\\

Test methods need to \textit{start} with \texttt{test}. You can do whatever you need to inside a test method, but importantly you should always \textbf{assert} something.

\subsection{Asserting}

``Asserting'' something just means saying ``This should be the case''. So if we assert that something is true, we're saying ``This value should be true''. An assertion can be correct, in which case the test passes, or incorrect, in which case the test fails.
\\

PHPUnit provides a \href{https://phpunit.readthedocs.io/en/9.0/assertions.html}{huge number of possible assertions} and Laravel \href{https://laravel.com/docs/master/http-tests#available-assertions}{provides some of its own too}, but for now we'll stick to some basics:

\begin{itemize}
    \item \texttt{assertTrue}: assert that something is \texttt{true}
    \item \texttt{assertSame}: assert that two things are the same type and value
\end{itemize}

First, let's make sure PHP knows which \texttt{Article} class to use. At the top of \texttt{ArticleTest}:

\begin{minted}{php}
    use App\Models\Article;
\end{minted}

Let's test our \texttt{truncate} method on our \texttt{Article} model by adding the following to \texttt{ArticleTest}:

\php{tests/Unit/ArticleTest.php}{06-unit-testing/figures/02-testTruncate}


Technically speaking you only really need \texttt{assertTrue}, but the other methods provide useful shorthands. We could use \texttt{assertSame} instead:

\begin{minted}{php}
    $this->assertSame("Blah blah blah", $article->truncate());
\end{minted}

\begin{minted}{php}
    $this->assertSame("Blah blah blah blah...", $article->truncate());
\end{minted}

Using \texttt{assertSame} is better, as if the test fails PHPUnit will tell you what it was expecting to happen. Notice that we put the \textbf{expected} value first and the \textbf{actual} value second, if it's the other way round the PHPUnit output will be backwards.
\\

Make sure to run your tests every time you make any changes:

\begin{minted}{bash}
    vendor/bin/phpunit --testsuite Unit
\end{minted}


\begin{infobox}{\texttt{Call to member function connection() on null}}
    If you're not doing anything involving databases this \textit{shouldn't} be necessary, but if you are testing a model and that model has a property that represents a \textit{date}, Eloquent needs to know what database type you are using. In order to do this we need to setup the Laravel app with all of its environment and configuration settings.
    \\

    To do this we need to extend the \textit{Laravel} \texttt{TestCase} class. Replace the \\ \texttt{use PHPUnit\textbackslash{}Framework\textbackslash{}TestCase} line with:

    \begin{minted}{php}
        use Tests\TestCase;
    \end{minted}

    Ideally we wouldn't need to do this, as it slows down our tests.
\end{infobox}



\subsection{\texttt{setUp}}

You might want to use the same \texttt{Article} object in multiple tests. We can set this up in the \texttt{setUp} method, which is automatically run before each test:

\php{tests/Unit/ArticleTest.php}{06-unit-testing/figures/03-setUp}


\section{Testing with a Database}

Sometimes we need to test things that involve the database. However, we don't want to pollute our normal database with test data. If we run the same test hundreds of times, we don't want to end up with the same test article hundreds of times. It's also harder to write tests if we don't know the state of the database before we start.
\\

For this reason we usually use a \textit{separate} database for testing. Laravel can be configured to use an \textbf{in-memory SQLite} database. You don't really need to worry about what that means as Laravel does all the work for us, but basically it's temporary and stored in memory (as opposed to on the hard-drive), so it's really fast.
\\

To use this separate test database edit your \texttt{phpunit.xml} file to uncomment the \texttt{DB\_CONNECTION} and \texttt{DB\_DATABASE} lines:

\begin{minted}{xml}
    ....
    <server name="CACHE_DRIVER" value="array"/>
    <!-- <server name="DB_CONNECTION" value="sqlite"/> -->
    <!-- <server name="DB_DATABASE" value=":memory:"/> -->
    <server name="MAIL_MAILER" value="array"/>
    ....
\end{minted}

Change to:

\begin{minted}{xml}
    ....
    <server name="CACHE_DRIVER" value="array"/>
    <server name="DB_CONNECTION" value="sqlite"/>
    <server name="DB_DATABASE" value=":memory:"/>
    <server name="MAIL_MAILER" value="array"/>
    ....
\end{minted}

\subsection{Using the Test Database in Tests}

We need to make a few changes to our test before we can work with the database.
\\

If you haven't already, you'll need to extend the \textit{Laravel} \texttt{TestCase} class. Replace the \texttt{use PHPUnit\textbackslash{}Framework\textbackslash{}TestCase} line with:

\begin{minted}{php}
    use Tests\TestCase;
\end{minted}

We can also tell Laravel to automatically refresh (clear and then migrate) the database for each test we run. This makes the database state more predictable.
\\

Add the following \texttt{use} statement:

\begin{minted}{php}
    use Illuminate\Foundation\Testing\RefreshDatabase;
\end{minted}

Then, as the very first line in your class add the following:

\begin{minted}{php}
    class ArticleTest extends TestCase
    {
        use RefreshDatabase;

        // ...rest of class code
    }
\end{minted}

This is called a \texttt{trait}: it effectively adds a method to your class.
\\

Now we can write tests involving the database.

\subsection{Writing Database Tests}

Now we can now write code that uses the database just as we normally would:

\php{tests/Unit/ArticleTest.php}{06-unit-testing/figures/04-testDatabase}

Because the database is wiped before each test, we can be sure that the first article in the database is the one we just added.
\\

The test above isn't particularly useful, but when it comes to testing our controllers it can be.

\begin{infobox}{\texttt{Cannot add a NOT NULL column with default value NULL}}
    SQLite doesn't allow you to add new columns to an existing database table without making them \texttt{nullable} or giving them a \texttt{default} value. If you get the error above when you run your tests, you'll need to update your migrations to either set a default (\texttt{->default(\$value)}) value or make the column nullable \\ (\texttt{->nullable()}).
\end{infobox}



\section{Additional Resources}

\begin{itemize}[leftmargin=*]
    \item \href{https://github.com/sebastianbergmann/phpunit}{PHPUnit}
    \item \href{http://laravel.com/docs/6.x/testing}{Laravel: Testing - Getting Started}
    \item \href{https://github.com/JeroenDeDauw/nyancat-phpunit-resultprinter}{Nyan Cat PHPUnit Result Printer}
\end{itemize}
