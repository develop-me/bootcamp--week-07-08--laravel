APIs are generally \textbf{stateless}, meaning that they do not use cookies or other methods to track who is making the request. As such, if we want the API to support users, every request made needs to include some way for the API to know who is making the request.
\\

APIs typically use \textbf{tokens} to authenticate users. A token is just a randomly generated string that is so long that you could never guess it by chance. They look like this:

\begin{minted}{text}
    MTQ0NjJkZmQ5OTM2NDE1ZTZjNGZmZjI3
\end{minted}

The most common protocol for using tokens is OAuth2 and looks like this:

\begin{enumerate}
    \item The user makes a request to a dedicated authorisation endpoint, providing credentials such as username and password
    \item The API checks whether these credentials are valid
    \item If valid, the API returns a token that is unique to the user
    \item The user makes all future requests supplying this token in the header
    \item The API checks the validity of the token
    \item The API grants or denies requests based on whether the user is valid and who the user is
\end{enumerate}

By using tokens the user doesn't have to send their password with every single request, which tends to make things more secure.

\pagebreak

\section{Passport}

Writing all of the code to deal with tokens is complicated\footnote{And involves a detailed knowledge of cryptography}, so we won't be doing it ourselves. Luckily there's a package, written by the same people as Laravel, that adds support for us.


\subsection{Setup}

First, install the Laravel Passport package:

\begin{minted}{bash}
    composer require laravel/passport
\end{minted}

Then run the Passport database migrations, which add tables to store access tokens (make sure you're in \texttt{vagrant ssh}):

\begin{minted}{bash}
    artisan migrate
\end{minted}

Next, run the Passport install script:

\begin{minted}{bash}
    artisan passport:install
\end{minted}

This will setup all of the cryptographic jiggerypokery that's required. Make a note of the \textbf{password grant} \texttt{Client ID} and \texttt{Client Secret} that are returned following successful installation. You should see something like this:

\begin{minted}{text}
    Password grant client created successfully.
    Client ID: 2
    Client secret: QtE4syo8QFwHE110mUMEj2zaJ3jgF3RYmF1JjZmU
\end{minted}

Next, we need to tell Laravel that our users can use API tokens. In the \texttt{User} model (\texttt{App/User.php}) add the \texttt{HasApiTokens} trait:

\phpinputminted{03/figures/01-HasApiTokens}


We'll also need to update \texttt{AuthServiceProvider.php} to use the Passport routes. This saves you having to write them yourself:

\phpinputminted{03/figures/02-AuthServiceProvider}


Finally, in \texttt{config/auth.php} config file, set the default ``guard'' to \texttt{api}:

\begin{minted}{php}
    'defaults' => [
        'guard' => 'api',
        // ...etc.
    ],
\end{minted}

And set the driver of the \texttt{api} authorisation guard to \texttt{passport}:

\begin{minted}{php}
    'guards' => [
        // ...etc.

        'api' => [
            'driver' => 'passport',
            'provider' => 'users',
        ],
    ],
\end{minted}

\subsection{Creating a User}

If you don't already have one you'll need to create a user.
\\

Run \texttt{tinker} (in \texttt{vagrant ssh} mode):

\begin{minted}{bash}
    artisan tinker
\end{minted}

Then use the \texttt{User} model's static \texttt{create} method to add a new user:

\begin{minted}{php}
    App\User::create([
        'name' => 'An Author',
        'email' => 'an.author@gmail.com',
        'password' => Hash::make('password'),
    ]);
\end{minted}

You should now be able to make a \texttt{POST} request in Postman to \\ \texttt{http://homestead.test/oauth/token} with the body:

\begin{minted}{json}
    {
        "grant_type": "password",
        "client_id": "<your_client_id>",
        "client_secret": "<your_client_secret>",
        "username": "an.author@gmail.com",
        "password": "password"
    }
\end{minted}

If everything is setup correctly, you should get back a token which you can use in future requests. As we're currently doing this manually, make sure you write down the token you're given.\footnote{As with most API stuff, authorisation with tokens is normally a fully programmatic affair, but it's useful to get a sense of what's going on}


\pagebreak


\section{Simple Authorisation}

For many use cases authorisation can be fairly simple: we just need to check if someone is logged in or not.
\\

We can use the provided Passport middleware to add authorisation to our routes.


\subsection{Adding Authorisation to a Single Route}

Add the \texttt{auth:api} middleware to any routes that require a valid access token:

\phpinputminted{03/figures/03-auth-route}

Try doing a \texttt{GET} request to \texttt{http://homestead.test/api/articles} in Postman (remember to add the header \texttt{Accept: application/json}). You should get a 401 response.
\\

Now choose \texttt{Bearer Token} in the Authorization tab and paste in the access token you got back from the \texttt{oauth/token} endpoint. Your request should now authenticate successfully and return the usual response.

\subsection{Adding Authorisation to Multiple Routes}

To add authorisation to an entire group:

\phpinputminted{03/figures/04-auth-route-group}



\section{Authorisation with Roles}

In reality our blog API probably needs different levels of authorisation. Posts should be open to anyone, only the owner of the blog should be able to write and edit blog posts, and only logged in users should be able to comment.
\\

This suggests two user roles: \textbf{Author} and \textbf{Subscriber}.


\subsection{Adding User Roles}

We'll need to modify the Users table to be able to store the user role.

\begin{minted}{bash}
    artisan make:migration modify_users_table
\end{minted}

In the generated migration file, add a \texttt{role} column in the \texttt{up()} method, and drop it in the \texttt{down()} method:

\phpinputminted{03/figures/05-ModifyUsersTable}

Make sure you migrate the database changes:

\begin{minted}{bash}
    artisan migrate
\end{minted}

Next, we'll need to update our \texttt{User} model to make this new column fillable.

\phpinputminted{03/figures/06-User}

Now let's update our existing user to have the role of \texttt{"author"} and generate a new user with the role \texttt{"subscriber"}.

\begin{minted}{bash}
    artisan tinker
\end{minted}

\begin{minted}{php}
    App\User::where('email', 'an.author@gmail.com')
            ->update(['role' => 'author']);

    App\User::create([
        'name' => 'A Subscriber',
        'role' => 'subscriber',
        'email' => 'a.subscriber@gmail.com',
        'password' => Hash::make('password')
    ]);
\end{minted}


\subsection{Checking User Roles}

Laravel provides a simple way for us to check who is making the request. In the \texttt{authorize()} method on the \texttt{FormRequest} classes that we extended, \texttt{\$this->user()} will return an instance of the \texttt{User} associated with the token sent with the request.
\\

Let's create a new request for \textit{storing} articles:

\begin{minted}{bash}
    artisan make:request ArticleStoreRequest
\end{minted}

The \texttt{rules()} method should return the same validation as \texttt{ArticleRequest}:

\begin{minted}{php}
    public function rules()
    {
        return [
            "title" => ["required", "string", "max:100"],
            "article" => ["required", "string"]
        ];
    }
\end{minted}

But let's write a conditional outcome in the \texttt{authorize()} method:

\phpinputminted{03/figures/07-authorize}

We'll need to tell the \texttt{store()} method in the \texttt{Articles} controller to use this new \texttt{Request}:

\phpinputminted{03/figures/08-Articles}


In Postman, make a \texttt{POST} request to \texttt{/oauth/token}, supplying the body as before, replacing the user details with the subscriber we generated above.
\\

Now, make a \texttt{POST} request to \texttt{/api/articles} supplying the returned subscriber access token as the Bearer Token. Your request should receive a \texttt{403} response. Then, replace the token with your author token. This should authenticate and a new article should be returned.

\section{Additional Resources}

\begin{itemize}[leftmargin=*]
    \item \href{https://laravel.com/docs/master/passport}{Laravel Passport}
    \item \href{https://www.digitalocean.com/community/tutorials/an-introduction-to-oauth-2}{An Introduction to OAuth 2}
\end{itemize}
