PHP is a server-side, dynamic scripting language, one of many (like ASP.NET, Ruby or even Python).
\\

Server-side means the code runs on a web server, which 'compiles' a page of HTML, as well as performing other actions when the code runs.
\\

Common things that server-side code can do, and often front-end or client-side cannot, are things like sending email, saving files, or interacting with a database.

\section{Web Servers}

Server-side languages, like PHP, run on a web server.
\\

PHP doesn't have to run this way, as you have experienced by running PHP scripts on the command line version of PHP, but to build apps that can be of use and "serve" other people, then we need a web server.
\\

\img{12cm}{01-server-side/img/centralised-architecture.png}{0em}{Centralised app architecture}

Common web servers are nginx (Engine Ex) and Apache on Linux or IIS on Windows (Internet Information Services).

\begin{infobox}{L.A.M.P.}
    The most common server 'stack' in the world running 72.5\% of websites:
    
    \item \textbf{Linux} operating system 
    \item \textbf{Apache} web server (or nginx)
    \item \textbf{MySQL} database (or MariaDB)
    \item \textbf{PHP} scripting language
\end{infobox}